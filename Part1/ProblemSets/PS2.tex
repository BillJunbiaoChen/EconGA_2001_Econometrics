\documentclass[11pt]{article}

%%METADATA
\title{GA2001 Econometrics \\Solution to Problem Set 2}
\author{
Junbiao Chen\thanks{E-mail: jc14076@nyu.edu.}
}
\date{\today}


%%PACKAGES
\usepackage{mdframed} % For boxed environments
\usepackage{graphicx}
\usepackage{grffile}
\usepackage{tabularx}
\usepackage{tikz}
\usepackage{setspace}
\usepackage{amsmath,amsthm,amssymb}
\usepackage[hyphens]{url}
\usepackage{natbib}
\usepackage[font=normalsize,labelfont=bf]{caption}
\usepackage[margin=1in]{geometry}
\usepackage{hyperref}
\hypersetup{colorlinks=true,urlcolor=blue,citecolor=blue}
\usepackage{stmaryrd}  %Package with \boxast command
\usepackage{enumerate}% http://ctan.org/pkg/enumerate %Supports lowercase Roman-letter enumeration
\usepackage{verbatim} %Package with \begin{comment} environment
%\usepackage{enumitem}
\usepackage{physics}
\usepackage{tikz}
\usepackage{listings}
\usepackage{upquote}
\usepackage{booktabs} %Package with \toprule and \bottomrule
\usepackage{etoc}     %Package with \localtableofcontents
\usepackage{placeins}    %Package that prevent repositioning the tables
\usepackage{multicol}
\usepackage{bm}
\usepackage{subfig}
\usepackage{csquotes}

\definecolor{dkgreen}{rgb}{0,0.6,0}
\definecolor{gray}{rgb}{0.5,0.5,0.5}
\definecolor{mauve}{rgb}{0.58,0,0.82}

\lstset{language=bash,
  frame=tb,
  aboveskip=3mm,
  belowskip=3mm,
  showstringspaces=false,
  columns=flexible,
  basicstyle={\small\ttfamily},
  numbers=none,
  numberstyle=\tiny\color{gray},
  keywordstyle=\color{blue},
  commentstyle=\color{dkgreen},
  stringstyle=\color{mauve},
  breaklines=true,
  breakatwhitespace=false,
  tabsize=3
}

\lstset{language=C,
  aboveskip=3mm,
  belowskip=3mm,
  showstringspaces=false,
  columns=flexible,
  basicstyle={\small\ttfamily},
  numbers=none,
  numberstyle=\tiny\color{gray},
  keywordstyle=\color{blue},
  commentstyle=\color{dkgreen},
  stringstyle=\color{mauve},
  breaklines=true,
  breakatwhitespace=false,
  tabsize=4
}

\definecolor{lightblue}{rgb}{0.68, 0.85, 0.9} %

%CUSTOM DEFINITIONS
\theoremstyle{definition}
\newtheorem{definition}{Definition}[section]
\newtheorem*{remark}{Remark}
\setcounter{secnumdepth}{3}
\usepackage{tikz}
\usetikzlibrary{arrows.meta}
\usetikzlibrary{automata, positioning, arrows, calc}

\tikzset{
	->,  % makes the edges directed
	>=stealth, % makes the arrow heads bold
	shorten >=2pt, shorten <=2pt, % shorten the arrow
	node distance=3cm, % specifies the minimum distance between two nodes. Change if n
	every state/.style={draw=blue!55,very thick,fill=blue!20}, % sets the properties for each ’state’ n
	initial text=$ $, % sets the text that appears on the start arrow
 }

%% PROPOSITION
% Define the Proposition environment
\newmdenv[
  innerleftmargin=10pt, 
  innerrightmargin=10pt,
  innertopmargin=10pt,
  innerbottommargin=10pt,
  linecolor=black, 
  linewidth=1pt,
  backgroundcolor=white, 
  roundcorner=5pt
]{propositionbox}

\newtheoremstyle{boldtitle} % Define a new theorem style
  {10pt} % Space above
  {10pt} % Space below
  {\itshape} % Body font
  {} % Indent amount
  {\bfseries} % Theorem head font
  {.} % Punctuation after theorem head
  { } % Space after theorem head
  {} % Theorem head spec

\theoremstyle{boldtitle} % Use the custom style
\newtheorem{proposition}{Proposition} % Define the proposition environment

% Redefine the proposition environment to use the box
\newenvironment{boxedproposition}[1][]
{\begin{propositionbox}\begin{proposition}[#1]}
{\end{proposition}\end{propositionbox}}

%%FORMATTING
\usepackage[bottom]{footmisc}
\onehalfspacing
\numberwithin{equation}{section}
\numberwithin{figure}{section}
\numberwithin{table}{section}
\bibliographystyle{../bib/aeanobold-oxford}


%main text
\begin{document}
\maketitle
\section*{Problem 1.}
\paragraph{1.} Take $S \in \{A \subseteq \Omega': f^{-1}(A) \in \mathcal{F}\}$,
we have $S^c = \Omega' \backslash S$, and $f^{-1} (\Omega'\backslash S) = \Omega \backslash f^{-1} (S) \in \mathcal{F}$. (Closed under complementation.)
Then, we can also show that 
$f^{-1}(\bigcup_{i \in \mathbb{N}} S_i) =  \bigcup_{i \in \mathbb{N}} f^{-1}(S_i) \in \mathcal{F}$. (Closed under countable union.)
This concludes that $\mathcal{P}$ is a $\sigma$-algebra.

\paragraph{2.} Since $\mathcal{F}$ is a sigma-algebra, 
$\sigma(\mathcal{F}) \subseteq \mathcal{F}$.
It follows that, for any $A \in \mathcal{F}$, we have $A \in \sigma(\mathcal{F})$,
therefore, $\mathcal{F}\backslash \mathcal{A}$-measurable implies $\mathcal{F}\backslash \mathcal{\sigma(A)}$-measurable.





\section*{Problem 2.} 
\paragraph{1.} By the definition of measurable functions, we have 
for any $A \in \mathcal{G}$, $f^{-1}(A) \in \mathcal{F}$, 
and for any $B \in \mathcal{H}$, $g^{-1}(B) \in \mathcal{G}$.
Now, consider a function $f^{-1}(g^{-1}(.))$, we have 
for any $B \in \mathcal{H}$, $f^{-1}(g^{-1}(B)) \in \mathcal{F}$ because $g^{-1}(B) \in \mathcal{G}$.
\(\blacksquare\)

\paragraph{2.}
Let $h: \mathcal{B}(\mathbb{R}^2) \rightarrow \mathcal{B}(\mathbb{R})$.
Since for each $B \in \mathcal{B}(\mathbb{R})$, we can find $h^{-1}(B) \in \mathcal{B}(\mathbb{R}^2)$,
$h$ is a $\mathcal{B}(\mathbb{R}^2)\backslash \mathcal{B}(\mathbb{R})$-measurable function.
By \textbf{1.}, we know that the composition of measurable functions is measurable.
Therefore, $\omega \circ h$ is $\mathcal{F}\backslash \mathcal{B}(\mathbb{R})$-measrable.
In conclusion, $f+g$ and $fg$ are Borel-measurable. \(\blacksquare\)


\paragraph{3.} Since $X$ is a random variable, we know that $X$ is a $\Omega \backslash \mathbb{R}$-measurable function.
Also, in the clase, we prove that 
\[
\mathcal{F}_X \subseteq \mathcal{F}_Y \text{ iff } \exists g: \mathbb{R}^{\text{dim} Y} \rightarrow \mathbb{R}^{\text{dim} X}
\]
Notice that $\mathbb{R}_{+} \subseteq \mathbb{R}$, there exists a function $g_+: \mathbb{R} \rightarrow \mathbb{R}_{+}$.
Since both $X$ and $g_+$ are measurable, by Problem 2. (1), we have $(X \circ g_+)$ is $\Omega \backslash \mathbb{R}_+$-measurable.
Thus, $X^+(\omega)$ is a random variable.
Similarily, we have $(X \circ g_-)$ is $\Omega \backslash \mathbb{R}_-$-measurable, where $g_-: \mathbb{R} \rightarrow \mathbb{R}_-$.
\(\blacksquare\)

\begin{figure}[htbp]
  \centering
  \begin{tikzpicture}[node distance=2.5cm, auto]

  % Nodes
  \node (omega) at (0, 0) {\( \Omega \)};
  \node (R) at (3, 0) {\( \mathbb{R} \)};
  \node (Rplus) at (6, 1.5) {\( \mathbb{R}_+ \)};
  \node (Rminus) at (6, -1.5) {\( \mathbb{R}_- \)};
  
  % Arrows
  \draw[->, thick] (omega) -- node[above] {\( X \)} (R);
  \draw[->, thick] (R) -- node[above right] {\( g_+ \)} (Rplus);
  \draw[->, thick] (R) -- node[below right] {\( g_- \)} (Rminus);

\end{tikzpicture}
\end{figure}

\section*{Problem 3.} 
\paragraph{1.} (a) ``Non-negativity'': $P_X (B) \geq 0$ for any $B \in \mathcal{B}(\mathbb{R})$ as the codomain is $[0,1]$.

(b) ``Countable-additivity'': Let $\{B_i\} \in \mathcal{B}(\mathbb{R})$ be a collection of disjoint sets.
$P_X (\sqcup B_i) = P(X^{-1} (\sqcup B_i) ) = P(\sqcup X^{-1} (B_i) ) = \sum_i P(X^{-1} (B_i) ) = \sum_i P_X(B_i)$.

(c) Finally, we have $P_X (\mathbb{R}) = P(X^{-1} (\mathbb{R})) = P(\Omega) = 1$.
The last equation, $P(\Omega) = 1$, holds because $P$ is a probability space.


\paragraph{2.} 
Let $s, t \in \mathbb{R}$ with $s > t$.
We have 
\begin{align*}
  F_X(s) - F_X(t) & = P_X(-\infty, s] - P_X(-\infty, t] \\ 
    & = P(X^{-1}(-\infty, s]) - P(X^{-1}(-\infty, t]) \\ 
    & = P(X^{-1}(t, s]) > 0
\end{align*}
Take $t \in \mathbb{R}$, consider a sequence of intervals such that 
$\{A_n: (-\infty, t + \frac{1}{n}) \}_{n=1}^\infty$.
Note that $A_n$ is decreasing, i.e., $A_1 \supset A_2 \supset A_3 \ldots$.
Therefore, we have $\lim_{x \rightarrow t^+} F_X(x) = F_X(\bigcap_{n=1}^\infty A_n) = F_X((-\infty, t]) = F_X(t)$.
\textbf{Thus, $F_X$ is right-continous}.

Take $m, t \in \mathbb{R}$ and $ m > t$, we have $F_X((\infty, m]) = F_X((\infty, t]) + F_X((t, m])$.
As $t \rightarrow -\infty$, $F_X((\infty, m]) = F_X((t, m])$,
therefore, 
\[
F_X((-\infty, t]) = F_X((\infty, m]) - F_X((t, m]) = 0.
\]

As $t \rightarrow \infty$, $F_X(t) = P_X(\mathbb{R}) = 1$.


\section*{Problem 4.} 
\paragraph{1.} Note that $\mathbb{E}(Y) = Y$ if $Y$ is constant.
If follows that $\mathbb{E}(XY) = Y \mathbb{E}(X) = \mathbb{E}(Y) \mathbb{E}(X)$.
Note that the first equality is due to definition of expectation.
Similarily, $\mathbb{E}(X=YY) = Y \mathbb{E}(Y) = \mathbb{E}(Y) \mathbb{E}(Y)$.
Therefore, if $Y$ is constant, then $X$ and $Y$ are independent.
$Y$ and $Y$ are independent. \(\blacksquare\)

\paragraph{2.} I apply Fubini's theorem to prove that if $X$ and $Y$ are 
independent then $\mathbb{E}(f(X)g(Y)) = \mathbb{E}(f(X)) \mathbb{E}(g(Y))$.

\textbf{Proof} 
\begin{align*}
  \mathbb{E}(f(X)g(Y)) & = \int_{\mathbb{R}^2} f(x)g(y) (\mu \times v) (dx \times dy) \\
  & = \int_{\mathbb{R}}\bigg(\int_{\mathbb{R}} f(x)g(y)\mu dx \bigg) v dy \\
  & = \int_{\mathbb{R}}g(y) \bigg(\int_{\mathbb{R}} f(x)\mu dx \bigg) v dy \\
  & = \int_{\mathbb{R}}g(y) \mathbb{E}(f(X)) v dy \\
  & = \mathbb{E}(f(X))  \int_{\mathbb{R}}g(y) v dy \\
  & = \mathbb{E}(f(X)) \mathbb{E}(g(Y)) 
\end{align*}
\(\blacksquare\)

%% =========== %% ========= %%
\bibliography{../bib/notes.bib}

\end{document}