\documentclass[11pt]{article}

%%METADATA
\title{GA2001 Econometrics (Part 2) \\Solution to Problem Set 4}
\author{
Junbiao Chen\thanks{E-mail: jc14076@nyu.edu.}
}
\date{\today}


%%PACKAGES
\usepackage{mdframed} % For boxed environments
\usepackage{graphicx}
\usepackage{grffile}
\usepackage{tabularx}
\usepackage{setspace}
\usepackage{amsmath,amsthm,amssymb}
\usepackage[hyphens]{url}
\usepackage{natbib}
\usepackage[font=normalsize,labelfont=bf]{caption}
\usepackage[margin=1in]{geometry}
\usepackage{hyperref}
\hypersetup{colorlinks=true,urlcolor=blue,citecolor=blue}
\usepackage{stmaryrd}  %Package with \boxast command
\usepackage{enumerate}% http://ctan.org/pkg/enumerate %Supports lowercase Roman-letter enumeration
\usepackage{verbatim} %Package with \begin{comment} environment
%\usepackage{enumitem}
\usepackage{physics}
\usepackage{tikz}
\usepackage{pgfplots}
\pgfplotsset{compat=1.17}
\usepackage{listings}
\usepackage{xcolor}
\usepackage{upquote}
\usepackage{booktabs} %Package with \toprule and \bottomrule
\usepackage{etoc}     %Package with \localtableofcontents
\usepackage{placeins}    %Package that prevent repositioning the tables
\usepackage{multicol}
\usepackage{bm}
\usepackage{subfig}
\usepackage{csquotes}

\definecolor{dkgreen}{rgb}{0,0.6,0}
\definecolor{gray}{rgb}{0.5,0.5,0.5}
\definecolor{mauve}{rgb}{0.58,0,0.82}

\lstset{language=bash,
  frame=tb,
  aboveskip=3mm,
  belowskip=3mm,
  showstringspaces=false,
  columns=flexible,
  basicstyle={\small\ttfamily},
  numbers=none,
  numberstyle=\tiny\color{gray},
  keywordstyle=\color{blue},
  commentstyle=\color{dkgreen},
  stringstyle=\color{mauve},
  breaklines=true,
  breakatwhitespace=false,
  tabsize=3
}

\lstset{language=C,
  aboveskip=3mm,
  belowskip=3mm,
  showstringspaces=false,
  columns=flexible,
  basicstyle={\small\ttfamily},
  numbers=none,
  numberstyle=\tiny\color{gray},
  keywordstyle=\color{blue},
  commentstyle=\color{dkgreen},
  stringstyle=\color{mauve},
  breaklines=true,
  breakatwhitespace=false,
  tabsize=4
}

% Define Julia style for lstlisting
\lstdefinelanguage{Julia}{
    morekeywords={
        for, end, in, if, else, elseif, while, break, continue, return, function,
        struct, mutable, begin, do, using, import, export, const, let, local, global,
        try, catch, finally, true, false, nothing, quote, macro, module, baremodule,
        where, abstract, typealias, type, bitstype
    },
    sensitive=true,
    morecomment=[l]{\#},
    morestring=[b]",
    morestring=[b]',
}

\lstset{
    language=Julia,
    basicstyle=\ttfamily\footnotesize,
    keywordstyle=\color{blue}\bfseries,
    commentstyle=\color{gray}\itshape,
    stringstyle=\color{red},
    numbers=left,
    numberstyle=\tiny\color{gray},
    stepnumber=1,
    numbersep=5pt,
    showspaces=false,
    showstringspaces=false,
    breaklines=true,
    breakatwhitespace=true,
    frame=single,
    captionpos=b
}


\definecolor{lightblue}{rgb}{0.68, 0.85, 0.9} %

%CUSTOM DEFINITIONS
\theoremstyle{definition}
\newtheorem{definition}{Definition}[section]
\newtheorem*{remark}{Remark}
\setcounter{secnumdepth}{3}
\usepackage{tikz}
\usetikzlibrary{arrows.meta}
\usetikzlibrary{automata, positioning, arrows, calc}

\tikzset{
	->,  % makes the edges directed
	>=stealth, % makes the arrow heads bold
	shorten >=2pt, shorten <=2pt, % shorten the arrow
	node distance=3cm, % specifies the minimum distance between two nodes. Change if n
	every state/.style={draw=blue!55,very thick,fill=blue!20}, % sets the properties for each ’state’ n
	initial text=$ $, % sets the text that appears on the start arrow
 }

%% PROPOSITION
% Define the Proposition environment
\newmdenv[
  innerleftmargin=10pt, 
  innerrightmargin=10pt,
  innertopmargin=10pt,
  innerbottommargin=10pt,
  linecolor=black, 
  linewidth=1pt,
  backgroundcolor=white, 
  roundcorner=5pt
]{propositionbox}

\newtheoremstyle{boldtitle} % Define a new theorem style
  {10pt} % Space above
  {10pt} % Space below
  {\itshape} % Body font
  {} % Indent amount
  {\bfseries} % Theorem head font
  {.} % Punctuation after theorem head
  { } % Space after theorem head
  {} % Theorem head spec

\theoremstyle{boldtitle} % Use the custom style
\newtheorem{proposition}{Proposition} % Define the proposition environment

% Redefine the proposition environment to use the box
\newenvironment{boxedproposition}[1][]
{\begin{propositionbox}\begin{proposition}[#1]}
{\end{proposition}\end{propositionbox}}

%%FORMATTING
\usepackage[bottom]{footmisc}
\onehalfspacing
\numberwithin{equation}{section}
\numberwithin{figure}{section}
\numberwithin{table}{section}
\bibliographystyle{../bib/aeanobold-oxford}


%main text
\begin{document}
\maketitle
\section{Problem 1. Hausman-Wu Test}
\subsection{a. Answer}
To show $P_Z P_{Z_1} = P_{Z_1}$, note that 
\[
Z_1 = Z \begin{bmatrix}
    I_{k1} \\ 0
\end{bmatrix}
\]
Hence, we have 
\begin{align*}
Z(Z'Z)^{-1}Z' Z_1(Z_1'Z_1)^{-1}Z_1' & = Z(Z'Z)^{-1} Z' Z \begin{bmatrix}
    I_{k1} \\ 0
\end{bmatrix} (Z_1'Z_1)^{-1}Z_1' \\ 
& = Z \begin{bmatrix}
    I_{k1} \\ 0
\end{bmatrix}(Z_1'Z_1)^{-1}Z_1'\\
& = Z_1(Z_1'Z_1)Z_1'
\end{align*}

\subsection{b and c. Answers}
Using CLT, we have 
\[
\sqrt{n}\left(\begin{bmatrix} \hat{\beta}_1 \\ \hat{\beta}_2 \end{bmatrix} 
- \begin{bmatrix} \beta  \\ \beta \end{bmatrix}  \right)
\stackrel{d}{\rightarrow} \mathcal{N}(0, V)
\]
where $V$ is given by 
\[
V_{11}
= \sigma_0^{2}\left(
\mathbb{E}\!\left[x_i z_{i1}'\right]\,
\mathbb{E}\!\left[z_{i1} z_{i1}'\right]^{-1}\,
\mathbb{E}\!\left[z_{i1} x_i'\right]
\right)^{-1}
\]
\[
V_{22}
= \sigma_0^{2}\left(
\mathbb{E}\!\left[x_i z_i'\right]\,
\mathbb{E}\!\left[z_i z_i'\right]^{-1}\,
\mathbb{E}\!\left[z_i x_i'\right]
\right)^{-1}
\]
\[
V_{12} = 
\sigma_0^{2}\left(
\mathbb{E}\!\left[x_i z_i'\right]\,
\mathbb{E}\!\left[z_i z_i'\right]^{-1}\,
\mathbb{E}\!\left[z_i x_i'\right]
\right)^{-1}
\]
\[
V_{21} = V_{12}' = V_{12}.
\]
Hence we have 
\[
V_{22} - V_{12} - V_{21} = -V_{22}.
\]

\subsection{d. Answer}
Applying Delta method to the result above, we have 
\[
\sqrt{n} (\hat{\beta}_1 - \hat{\beta}_2) \stackrel{d}{\rightarrow} 
\mathcal{N} \left(0,\ V_{11}+V_{22}-V_{12}-V_{21}\right) 
= 
\mathcal{N}(0, V_{11} - V_{22}).
\]
\textbf{Remark:} Under the null hypothesis that both estimators are consistent 
and under regularity conditions ensuring a joint CLT,
the Hausman-Wu statistic follows $\chi^2$ distribution.
\[
H \;=\; n(\hat{\beta}_1-\hat{\beta}_2)'\,(V_{11}-V_{22})^{-1}\,(\hat{\beta}_1-\hat{\beta}_2)
\ \xrightarrow{d}\ \chi^2_{k},
\]
where $k=\dim(\beta)$.


\section{Problem 2. Abadie (2003)'s $\kappa$ function.}
\subsection{a. solution}
begin{table}[htbp]
    \centering
    \begin{tabular}{lll}
        \toprule
        & \textbf{$Z_i$ = 0} & \textbf{$Z_i$ = 1} \\ \midrule
         $D_i$ = 0 &  Never taker &  \\
        \( k_{max} \) & \( 1.05 \cdot k_{ss} \) \\
        \( k_{vec} \) & \( k_{min} : \frac{k_{max} - k_{min}}{I-1} : k_{max} \) \\
        \( \text{max\_iter} \) & \( 1000 \) (maximum number of iterations) \\
        \( \text{tol} \) & \( 10^{-8} \) (tolerance for convergence) \\ \bottomrule
    \end{tabular}
    \caption{Summary of Parameters}
\end{table}




\section{Problem 3. Weak and Many Instrumental Variables}
\subsection{a. Answer}
\begin{figure}[htbp]
  \centering
  \includegraphics[width=0.9\textwidth]{weak_iv/images/comparison_ols_iv.png}
  \caption{OLS vs. IV under weak/many intruments}
\end{figure}


%% =========== %% ========= %%
\bibliography{../bib/notes.bib}

\end{document}