\documentclass[11pt]{article}

%%METADATA
\title{GA2001 Econometrics \\Solution to Problem Set 1}
\author{
Junbiao Chen\thanks{E-mail: jc14076@nyu.edu.}
}
\date{\today}


%%PACKAGES
\usepackage{mdframed} % For boxed environments
\usepackage{graphicx}
\usepackage{grffile}
\usepackage{tabularx}
\usepackage{setspace}
\usepackage{amsmath,amsthm,amssymb}
\usepackage[hyphens]{url}
\usepackage{natbib}
\usepackage[font=normalsize,labelfont=bf]{caption}
\usepackage[margin=1in]{geometry}
\usepackage{hyperref}
\hypersetup{colorlinks=true,urlcolor=blue,citecolor=blue}
\usepackage{stmaryrd}  %Package with \boxast command
\usepackage{enumerate}% http://ctan.org/pkg/enumerate %Supports lowercase Roman-letter enumeration
\usepackage{verbatim} %Package with \begin{comment} environment
%\usepackage{enumitem}
\usepackage{physics}
\usepackage{tikz}
\usepackage{listings}
\usepackage{upquote}
\usepackage{booktabs} %Package with \toprule and \bottomrule
\usepackage{etoc}     %Package with \localtableofcontents
\usepackage{placeins}    %Package that prevent repositioning the tables
\usepackage{multicol}
\usepackage{bm}
\usepackage{subfig}
\usepackage{csquotes}

\definecolor{dkgreen}{rgb}{0,0.6,0}
\definecolor{gray}{rgb}{0.5,0.5,0.5}
\definecolor{mauve}{rgb}{0.58,0,0.82}

\lstset{language=bash,
  frame=tb,
  aboveskip=3mm,
  belowskip=3mm,
  showstringspaces=false,
  columns=flexible,
  basicstyle={\small\ttfamily},
  numbers=none,
  numberstyle=\tiny\color{gray},
  keywordstyle=\color{blue},
  commentstyle=\color{dkgreen},
  stringstyle=\color{mauve},
  breaklines=true,
  breakatwhitespace=false,
  tabsize=3
}

\lstset{language=C,
  aboveskip=3mm,
  belowskip=3mm,
  showstringspaces=false,
  columns=flexible,
  basicstyle={\small\ttfamily},
  numbers=none,
  numberstyle=\tiny\color{gray},
  keywordstyle=\color{blue},
  commentstyle=\color{dkgreen},
  stringstyle=\color{mauve},
  breaklines=true,
  breakatwhitespace=false,
  tabsize=4
}

\definecolor{lightblue}{rgb}{0.68, 0.85, 0.9} %

%CUSTOM DEFINITIONS
\theoremstyle{definition}
\newtheorem{definition}{Definition}[section]
\newtheorem*{remark}{Remark}
\setcounter{secnumdepth}{3}
\usepackage{tikz}
\usetikzlibrary{arrows.meta}
\usetikzlibrary{automata, positioning, arrows, calc}

\tikzset{
	->,  % makes the edges directed
	>=stealth, % makes the arrow heads bold
	shorten >=2pt, shorten <=2pt, % shorten the arrow
	node distance=3cm, % specifies the minimum distance between two nodes. Change if n
	every state/.style={draw=blue!55,very thick,fill=blue!20}, % sets the properties for each ’state’ n
	initial text=$ $, % sets the text that appears on the start arrow
 }

%% PROPOSITION
% Define the Proposition environment
\newmdenv[
  innerleftmargin=10pt, 
  innerrightmargin=10pt,
  innertopmargin=10pt,
  innerbottommargin=10pt,
  linecolor=black, 
  linewidth=1pt,
  backgroundcolor=white, 
  roundcorner=5pt
]{propositionbox}

\newtheoremstyle{boldtitle} % Define a new theorem style
  {10pt} % Space above
  {10pt} % Space below
  {\itshape} % Body font
  {} % Indent amount
  {\bfseries} % Theorem head font
  {.} % Punctuation after theorem head
  { } % Space after theorem head
  {} % Theorem head spec

\theoremstyle{boldtitle} % Use the custom style
\newtheorem{proposition}{Proposition} % Define the proposition environment

% Redefine the proposition environment to use the box
\newenvironment{boxedproposition}[1][]
{\begin{propositionbox}\begin{proposition}[#1]}
{\end{proposition}\end{propositionbox}}

%%FORMATTING
\usepackage[bottom]{footmisc}
\onehalfspacing
\numberwithin{equation}{section}
\numberwithin{figure}{section}
\numberwithin{table}{section}
\bibliographystyle{../bib/aeanobold-oxford}


%main text
\begin{document}
\maketitle
\section*{Problem 1.}
\subsection*{(a) The Probability Space of an Urns-Balls Experiment} 
The probability space for an $n$-urns-$k$-balls experiement with $n > k$ is given by 
\begin{itemize}
  \item $\Omega = \{(x_1, x_2, \dots, x_n): \sum_{i=1}^n x_i = k \text{ and } x_i \in \mathbb{Z} \}$ with $|\Omega| = \binom{k+n-1}{n-1}$.
  \item $\mathcal{F} = 2^{\Omega}$, the power set of $\Omega$.
  % \item \begin{align*}
  %     \mathcal{F} = \{ & \emptyset, \\
  %                     & \{(x_1, x_2, \dots, x_n): x_i = k \text{ and }  x_{j \neq i} = 0 \text{ and } x_i \in \mathbb{Z}^{++} \}, \\
  %                     \dots \\
  %                     & \{(x_1, x_2, \dots, x_n): \underbrace{x_{i_1} + \dots + x_{i_m}}_{m} = k \text{ and others } = 0 \text{ and } x_{i_1}, \dots, x_{i_m} \in \mathbb{Z}^{++} \}, \\
  %                     \dots \\
  %                     & \{(x_1, x_2, \dots, x_n): \sum_{i=1}^n x_i = k \text{ and } x_i \in \mathbb{Z}^{++} \}
  %                     \}
  % \end{align*}
  % \item \begin{align*}
  %   & P(\emptyset) =  0; \\
  %       & P(A_1) = \frac{n}{\binom{k+n-1}{n-1}}, \text{where } A_1 = \{(x_1, x_2, \dots, x_n): x_i = k \text{ and }  x_{j \neq i} = 0 \text{ and } x_i \in \mathbb{Z}^{++} \}; \\
  %       \dots \\
  %       & P(A_m) = \frac{\binom{n}{m} \binom{k-1}{m-1}}{\binom{k+n-1}{n-1}}, \text{where } A_1 = \{(x_1, x_2, \dots, x_n): \underbrace{x_{i_1} + \dots + x_{i_m}}_{m} = k \text{ and others } = 0 \text{ and } x_{i_1}, \dots, x_{i_m} \in \mathbb{Z}^{++} \}; \\
  %       \dots \\
  %       & P(A\{(x_1, x_2, \dots, x_n): \sum_{i=1}^n x_i = k \text{ and } x_i \in \mathbb{Z}^{++} \}) = 1
  % \end{align*}
  \item For any event $A \in \mathcal{F}$, $P(A) = \frac{|A|}{\binom{k+n-1}{n-1}}$, where $|A|$ is the cardinality of set $A$.
\end{itemize}

\subsection*{(b) The Law of Total Probability}
Since $(B_i)_{i \in \mathbb{N}}$ is a partition of $\Omega$,
$(A \bigcap B_i)$ are disjoint across $i$.
Therefore, $\sum_{i = 1}^\infty P(A \bigcap B_i) = P(\cup_i (A \bigcap B_i)) = P(A)$. \(\blacksquare\)


\subsection*{(c)} 
Denote $P(B_i)$ the probability that exactly $i$ urns are empty.
Then the probability of all empty urns are located to the left of non-empty urns is given by 
\[
P(A) = \frac{1}{\binom{k+n-1}{n-1}} \left(\sum_{i=1}^{k} \frac{1}{P(B_{n - i})} \right)
\]


\section*{Problem 2.}
Let $\mathcal{I} := \{(a, b); a, b \in \mathbb{R}\}$ and $\mathcal{F} = \sigma (\mathcal{I})$.
\subsection*{(a)} 
Note that $(c, d] = \bigcap_{n}^\infty (c, d+\frac{1}{n})$. 
Since $(c, d+\frac{1}{n}) \in \mathcal{F}$, and the countable intersection is closed in $\sigma$-algebra,
$
\{(c, d]\} \subseteq \mathcal{F}
$
We need to show that $\mathcal{F}$ is strictly larger than $\{(c, d]: c < d; a, b \in \mathbb{R} \}$.
In particular, $[a,b) \in \mathcal{F}$ (because $[a,b) = \cup_{n=1}^\infty (a - \frac{1}{n},b)$),
while $[a,b)$ cannot be generated by $(c,d]$ with either countable union or intersection.
Therefore, we have $\{(c, d]: c < d; a, b \in \mathbb{R} \} \subset \mathcal{F}$.


Similarly, $(a, \infty) = \cup_{n}^\infty (a, n)$,
therefore, $\{(a, \infty)\} \subseteq \mathcal{F}$.
However, $\{b\} \in \mathcal{F}$ while it cannot be generated by $(a, \infty)$ with either countable union or intersection.
Thus, $\{(a, \infty): a \in \mathbb{R} \} \subset \mathcal{F}$. \(\blacksquare\)

\subsection*{(b)} 
To show that both $\mathcal{A} = \{[a, b]: a < b; a, b \in \mathbb{R} \}$ and $\mathcal{B} = \{(-\infty, b]: b \in \mathbb{R} \}$ generate $\mathcal{F}$,
we need to show that $\sigma(\mathcal{A}) = \mathcal{F}$ and $\sigma(\mathcal{B}) = \mathcal{F}$.
\paragraph{Part 1.}
First, we show that $\mathcal{I} \subseteq \sigma(\mathcal{A})$.
Pick any $(a, b) \in \sigma(\mathcal{I})$,
\begin{align*}
  (a, b) = \cup^\infty [a+\frac{1}{n}, b-\frac{1}{n}] \in \sigma(\mathcal{A})
\end{align*}
Therefore, $\sigma(\mathcal{I}) \subseteq \sigma(\mathcal{A})$.

Pick any $[a, b] \in \sigma(\mathcal{A})$,  
one can show that 
$[a, b] = \{a\} \cup (a, b) \cup \{b\} \in \sigma(\mathcal{I})$.%
\footnote{
  $\{a\} \in \sigma(\mathcal{I})$ because $\{a\} = \bigcap_{n=1}^\infty (a, a+\frac{1}{n})$.
}
Therefore,
$[a, b] \in \sigma(\mathcal{I})$. 
Thus, $\sigma(\mathcal{A}) \subseteq \sigma(\mathcal{I})$.
It concludes that, $\sigma(\mathcal{I}) = \sigma(\mathcal{A})$. 

\paragraph{Part 2.}
Pick $(-\infty, b] \in \sigma (\mathcal{B})$, it follows that 
\begin{align*}
  (-\infty, b] & = \mathbb{R}\backslash (b, \infty) \in \mathcal{F} \\
\Rightarrow & \sigma (\mathcal{B}) \subseteq \mathcal{F}
\end{align*}
%
Pick $(a, b) \in \mathcal{F}$, it follows that 
\begin{align*}
  (a, b) & = \mathbb{R}\backslash (-\infty, a] \bigcap \bigg( \cup_{n=1}^\infty  (-\infty, b - \frac{1}{n}] \bigg) \in \sigma(\mathcal{B}) \\
\Rightarrow & \mathcal{F} \subseteq \sigma (\mathcal{B}) 
\end{align*}
Therefore, $ \mathcal{F} = \sigma (\mathcal{B}) $.

\subsection*{(c)}
False. Because this set doesn't include singleton.


\section*{Problem 3.}
The set $\mathcal{L} := \{(x, y) \in \mathbb{R}^2: x = y; 0 \leq x \leq 1 \}$ can be obtained by countable intersection of open sets in $\mathbb{R}^2$.
Concretely, 
\begin{align*}
\mathcal{L} = & \bigcap_{n \in \mathbb{N}} \{ (x, y) \in \mathbb{R}^2: y > x + \frac{1}{n} \text{ and } x \in (-\frac{1}{n}, 1 + \frac{1}{n})\} \bigcap \\ 
  & \bigcap_{n \in \mathbb{N}} \{ (x, y) \in \mathbb{R}^2: y < x - \frac{1}{n} \text{ and } x \in (-\frac{1}{n}, 1 + \frac{1}{n})\} 
\end{align*}


\section*{Problem 4.}
Consider the following sequence:
\begin{align*}
  A_1 & = \{1, 2, \ldots,  \} \\
  A_2 & = \{2, 3, \ldots,  \} \\
  ...
\end{align*}
It follows that $\bigcap_{i \in \mathbb{N}} A_i = \emptyset$ (because suppose $n \in \mathbb{N}$ is in 
$A_n$, there exists $A_{m}$, s.t. $n \notin A_m$ for $i \geq m$.)
Now we use length measure.
Therefore, we have $\mu(\bigcap_{i \in \mathbb{N}} A_i) = 0$ while 
$\lim_{i \rightarrow \infty} \mu(A_i) = \infty$. (In fact, $\mu(A_i) = \infty, \forall i$.)

\section*{Problem 5.}
Recall that in $(\mathbb{R}^k, \mathbb{B}^k)$, the Lebesgue measure $\lambda$ is 
\begin{align*}
  \lambda(B) \equiv \inf_{B \subseteq \bigcup_{j=1}^\infty \{\times_{i=1}^k (a_{ij}, b_{ij}) \} } \sum_{j=1}^\infty \pi_{j=1}^k (b_{ji} - a_{ij})
\end{align*}
\paragraph{(a.)} Since $a-a = 0$, we have $\lambda(\{a\}) = 0$
\paragraph{(b.)} Since every countable set $A$ can be enumerated using singletons,
$\mu(A) = 0$ for every countable set $A \in \mathcal(B)(\mathbb{R})$. 
\paragraph{(C.)} Similar to \textbf{(b.)}, $\lambda((a, b)) = \lambda([a, b]) - \lambda(\{a\}) -\lambda(\{b\}) = b - a - 0 -0$.
Following this logic, $\lambda([a, b)) = \lambda((a, b])  = b -a$.


\section*{Problem 6.}
Note that $A \Delta B = (A \backslash B) \cup (B \backslash A)$. 
Since $(A \backslash B) \cap (B \backslash A) = \emptyset$, 
$P\bigg((A \backslash B) \cup (B \backslash A) \bigg) = P(A \backslash B) + P(B \backslash A)$.
\begin{align*}
  |P(A) - P(B)|& = |P\bigg((A\backslash B)  \cup (A \cap B) \bigg) - P\bigg((B\backslash A)  \cup (A \cap B) \bigg) | \\
  & = |P(A\backslash B) + P(A \cap B) - P(B\backslash A) - P(A \cap B)| \\
  & = |P(A\backslash B) - P(B\backslash A)| \\
  & \leq |P(A\backslash B) + P(B\backslash A)| = |P(A\Delta B)|
\end{align*}
\(\blacksquare\)
%% =========== %% ========= %%
\bibliography{../bib/notes.bib}

\end{document}