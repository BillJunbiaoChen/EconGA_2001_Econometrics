\documentclass[11pt]{article}

%%METADATA
\title{GA2001 Econometrics \\Solution to Problem Set 4}
\author{
Junbiao Chen\thanks{E-mail: jc14076@nyu.edu.}
}
\date{\today}


%%PACKAGES
\usepackage{mdframed} % For boxed environments
\usepackage{graphicx}
\usepackage{grffile}
\usepackage{tabularx}
\usepackage{tikz}
\usepackage{setspace}
\usepackage{amsmath,amsthm,amssymb}
\usepackage[hyphens]{url}
\usepackage{natbib}
\usepackage[font=normalsize,labelfont=bf]{caption}
\usepackage[margin=1in]{geometry}
\usepackage{hyperref}
\hypersetup{colorlinks=true,urlcolor=blue,citecolor=blue}
\usepackage{stmaryrd}  %Package with \boxast command
\usepackage{enumerate}% http://ctan.org/pkg/enumerate %Supports lowercase Roman-letter enumeration
\usepackage{verbatim} %Package with \begin{comment} environment
%\usepackage{enumitem}
\usepackage{physics}
\usepackage{tikz}
\usepackage{listings}
\usepackage{upquote}
\usepackage{booktabs} %Package with \toprule and \bottomrule
\usepackage{etoc}     %Package with \localtableofcontents
\usepackage{placeins}    %Package that prevent repositioning the tables
\usepackage{multicol}
\usepackage{bm}
\usepackage{subfig}
\usepackage{csquotes}

\definecolor{dkgreen}{rgb}{0,0.6,0}
\definecolor{gray}{rgb}{0.5,0.5,0.5}
\definecolor{mauve}{rgb}{0.58,0,0.82}

\lstset{language=bash,
  frame=tb,
  aboveskip=3mm,
  belowskip=3mm,
  showstringspaces=false,
  columns=flexible,
  basicstyle={\small\ttfamily},
  numbers=none,
  numberstyle=\tiny\color{gray},
  keywordstyle=\color{blue},
  commentstyle=\color{dkgreen},
  stringstyle=\color{mauve},
  breaklines=true,
  breakatwhitespace=false,
  tabsize=3
}

\lstset{language=C,
  aboveskip=3mm,
  belowskip=3mm,
  showstringspaces=false,
  columns=flexible,
  basicstyle={\small\ttfamily},
  numbers=none,
  numberstyle=\tiny\color{gray},
  keywordstyle=\color{blue},
  commentstyle=\color{dkgreen},
  stringstyle=\color{mauve},
  breaklines=true,
  breakatwhitespace=false,
  tabsize=4
}

\definecolor{lightblue}{rgb}{0.68, 0.85, 0.9} %

%CUSTOM DEFINITIONS
\theoremstyle{definition}
\newtheorem{definition}{Definition}[section]
\newtheorem*{remark}{Remark}
\setcounter{secnumdepth}{3}
\usepackage{tikz}
\usetikzlibrary{arrows.meta}
\usetikzlibrary{automata, positioning, arrows, calc}

\tikzset{
	->,  % makes the edges directed
	>=stealth, % makes the arrow heads bold
	shorten >=2pt, shorten <=2pt, % shorten the arrow
	node distance=3cm, % specifies the minimum distance between two nodes. Change if n
	every state/.style={draw=blue!55,very thick,fill=blue!20}, % sets the properties for each ’state’ n
	initial text=$ $, % sets the text that appears on the start arrow
 }

%% PROPOSITION
% Define the Proposition environment
\newmdenv[
  innerleftmargin=10pt, 
  innerrightmargin=10pt,
  innertopmargin=10pt,
  innerbottommargin=10pt,
  linecolor=black, 
  linewidth=1pt,
  backgroundcolor=white, 
  roundcorner=5pt
]{propositionbox}

\newtheoremstyle{boldtitle} % Define a new theorem style
  {10pt} % Space above
  {10pt} % Space below
  {\itshape} % Body font
  {} % Indent amount
  {\bfseries} % Theorem head font
  {.} % Punctuation after theorem head
  { } % Space after theorem head
  {} % Theorem head spec

\theoremstyle{boldtitle} % Use the custom style
\newtheorem{proposition}{Proposition} % Define the proposition environment

% Redefine the proposition environment to use the box
\newenvironment{boxedproposition}[1][]
{\begin{propositionbox}\begin{proposition}[#1]}
{\end{proposition}\end{propositionbox}}

%%FORMATTING
\usepackage[bottom]{footmisc}
\onehalfspacing
\numberwithin{equation}{section}
\numberwithin{figure}{section}
\numberwithin{table}{section}
\bibliographystyle{../bib/aeanobold-oxford}


%main text
\begin{document}
\maketitle
\section*{Problem 1.}
\paragraph{Proof} 
Since $\phi \neq \psi \Leftrightarrow (\phi > \psi) \cup (\psi > \phi)$. 
For $\phi > \psi$, denote $X_1 = \phi - \psi$, since $X_1$ is non-negative, we can invoke the 
claim proved in exercise 3 to get that 
\[
X_1 = 0 \text{ a.s.}
\]
Similarly, for $\psi > \phi$, denote $X_2 = \psi - \phi$, we can also get that
\[
X_2 = 0 \text{ a.s.}
\]
Therefore, we conclude that proof that 
\[
\int_A \phi d \mathbb{P} = \int_A \psi d \mathbb{P} \Rightarrow \phi = \psi \text{ a.s.}
\]
\(\blacksquare\)

\section*{Problem 2.}
\paragraph{1.} By Chebyshev inequality, we have 
\[
\mathbb{P}(|S_n - S| > \epsilon) \leq \frac{\mathbb{E}|S_n - S|^r}{\epsilon^r}
\]
Since $\frac{\mathbb{E}|S_n - S|^r}{\epsilon^r} \rightarrow 0$, we have $\mathbb{P}(|S_n - S|) \rightarrow 0$.
This completes the proof that $S_n \xrightarrow{P} S$.

\paragraph{2.} 
Given $\bar{X}_n = \frac{1}{n} \sum_{i=1}^n X_i$ and $\mathbb{E}(X) = \mu$, we have 
\begin{align*}
\mathbb{E}|\frac{1}{n} \sum_{i=1}^n (X_i - \mu)|^r & = \frac{1}{n^r} \mathbb{E}|\sum_{i=1}^n (X_i - \mu)|^r \\ 
& \leq \frac{1}{n^r} \mathbb{E}\sum_{i=1}^n |(X_i - \mu)|^r \\ 
& \leq \mathbb{E} \bigg[\max_i |X_i - \mu| \bigg] < \infty
\end{align*}
It follows that $ \frac{1}{n} \sum_{i=1}^n X_i \xrightarrow{P} \mu$.
Similarly, we have 
\begin{align*}
\mathbb{E}|\frac{1}{n^{1/2}} \sum_{i=1}^{n^{1/2}} (X_i - \mu)|^r & = \frac{1}{n^{r/2}} \mathbb{E}|\sum_{i=1}^{n^{1/2}} (X_i - \mu)|^r \\ 
& \leq \frac{1}{n^{r/2}} \mathbb{E}\sum_{i=1}^{n^{1/2}} |(X_i - \mu)|^r \\ 
& \leq \mathbb{E} \bigg[\max_i |X_i - \mu| \bigg] < \infty
\end{align*}
It follows that $\frac{1}{n^{1/2}} \sum_{i=1}^{n^{1/2}} X_i \xrightarrow{P} \mu$.
Also, $\frac{1}{\ln (n)} \sum_{i=1}^{\ln (n)} X_i \xrightarrow{P} \mu$.

\textbf{Remark:} We prefer $\bar{X}_n = \frac{1}{n} \sum_{i=1}^n X_i$ because 
$\sqrt(n) (\bar{X}_n - \mu) \xrightarrow{d} N(0, \sigma^2)$ per CLT.


\paragraph*{3.}
\textbf{Proof}
Because
\[
\operatorname{Var}(Y_i)=\mathbb E Y_i^{2}-(\mathbb EY_i)^{2}\le\mathbb E Y_i^{2}\le M,
\]
we have $\sigma_i^{2}:=\operatorname{Var}(Y_i)\le M$ for every $i$.
Since $Y_i$ are independent,
\[
\operatorname{Var}\left(\sum_{i=1}^{n}(Y_i)\right)
      =\sum_{i=1}^{n}\sigma_i^{2}.
\]
Hence
\[
\operatorname{Var}\!\Bigl(\frac1n\sum_{i=1}^{n}(Y_i-\mathbb EY_i)\Bigr)
      =\frac1{n^{2}}\sum_{i=1}^{n}\sigma_i^{2}
      \le\frac{nM}{n^{2}}
      \rightarrow 0.
\]
It follows that
\[
\mathbb E\!\Bigl[\Bigl(\frac1n\sum_{i=1}^{n}(Y_i-\mathbb EY_i)\Bigr)^{2}\Bigr],
\]
by the result proved in part 1,  
$\frac1n\sum_{i=1}^{n}(Y_i-\mathbb EY_i)\xrightarrow{\mathbb P}0$.
\(\blacksquare\)


\section*{Problem 3.}
\subsection*{1.} For $X_n = \text{Bern}(\frac{1}{n})$ and $X = \text{Bern}(0)$.
$X_n \xrightarrow{P} X$ because of strong law of large number.

\subsection*{2.} For $X_n = \text{Unif}(\frac{1}{n}, \frac{2}{n}, \ldots, \frac{n-1}{n}, 1)$ and $X = \text{U}[0,1]$.
$X_n \xrightarrow{d} X$ because of weak law of large number.


\section*{Problem 4.}
\subsection*{1.}
\paragraph{Proof}
Given the functional form of $g(X)$, the objective function is given by 
\[
\min_{a, b} \mathbb{E}\left[(Y - (a + b X))^2\right]
\]
The F.O.C with respect to $a$ and $b$ are  
\[
0 = - 2 \mathbb{E}\left[(Y -a - bX)\right]
\]

\[
0 = - 2 \mathbb{E}\left[X (Y -a - bX)\right]
\]
Then we have $\hat{a} = \mathbb{E}[Y] - \hat{b}\mathbb{E}[X]$, and 
\begin{align*}
0 & = \mathbb{E}\left[X (Y - \mathbb{E}[Y] + \hat{b}\mathbb{E}[X] - \hat{b}X)\right] \\ 
 & = \mathbb{E}\left[X (Y - \mathbb{E}[Y] + \hat{b}(\mathbb{E}[X] - X)) \right] 
\end{align*}
\[
\Rightarrow
\]

\[
\hat{b} = \frac{\mathbb{E}\left[X (Y - \mathbb{E}[Y] )\right] 
    }{
        \mathbb{E}\left[X(X - \mathbb{E}[X] ) \right] 
    } = \frac{\text{Cov}(X, Y)}{\text{Var}(X)}
\]
\(\blacksquare\)

\subsection*{2.} Suppose function $f: \{0, 1 \} \rightarrow \mathbb{R}$ is given by 
\[
f(x) = 
\begin{cases}
    \alpha \quad \text{ if } x = 0 \\
    \beta \quad  \text{ if } x = 1
\end{cases}
\]
which can be expressed as 
\[
\tilde{f}(x) = \alpha + (\beta - \alpha) x
\]

\subsection*{3.}
Given $\mathbb{P}\{X = 1\} = \mathbb{P}\{X = 0\} = \frac{1}{2}$, $\mathbb{E}[Y]= 0$, and $\mathbb{E}[XY]= 1$
we have 
\[
\mathbb{E}[X] = \frac{1}{2}, \quad \text{Var}(X) = \frac{1}{4}, \quad \text{Cov}(X, Y) = 1
\]
Therefore, $\beta^* = 4$ and $\alpha^* = -2$.
\[
\mathbb{E}[Y|X] = \alpha^* + \beta^* X = -2 + 4X
\].

\bibliography{../bib/notes.bib}

\end{document}